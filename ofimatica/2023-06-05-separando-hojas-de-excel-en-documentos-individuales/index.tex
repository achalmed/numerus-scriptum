\documentclass[
  jou,
  floatsintext,
  longtable,
  a4paper,
  nolmodern,
  notxfonts,
  notimes,
  colorlinks=true,linkcolor=blue,citecolor=blue,urlcolor=blue]{apa7}

\usepackage{amsmath}
\usepackage{amssymb}



\usepackage[bidi=default]{babel}
\babelprovide[main,import]{spanish}


\babelfont{rm}[,RawFeature={fallback=mainfontfallback}]{Latin Modern
Roman}
% get rid of language-specific shorthands (see #6817):
\let\LanguageShortHands\languageshorthands
\def\languageshorthands#1{}

\RequirePackage{longtable}
\RequirePackage{threeparttablex}

\makeatletter
\renewcommand{\paragraph}{\@startsection{paragraph}{4}{\parindent}%
	{0\baselineskip \@plus 0.2ex \@minus 0.2ex}%
	{-.5em}%
	{\normalfont\normalsize\bfseries\typesectitle}}

\renewcommand{\subparagraph}[1]{\@startsection{subparagraph}{5}{0.5em}%
	{0\baselineskip \@plus 0.2ex \@minus 0.2ex}%
	{-\z@\relax}%
	{\normalfont\normalsize\bfseries\itshape\hspace{\parindent}{#1}\textit{\addperi}}{\relax}}
\makeatother




\usepackage{longtable, booktabs, multirow, multicol, colortbl, hhline, caption, array, float, xpatch}
\usepackage{subcaption}


\renewcommand\thesubfigure{\Alph{subfigure}}
\setcounter{topnumber}{2}
\setcounter{bottomnumber}{2}
\setcounter{totalnumber}{4}
\renewcommand{\topfraction}{0.85}
\renewcommand{\bottomfraction}{0.85}
\renewcommand{\textfraction}{0.15}
\renewcommand{\floatpagefraction}{0.7}

\usepackage{tcolorbox}
\tcbuselibrary{listings,theorems, breakable, skins}
\usepackage{fontawesome5}

\definecolor{quarto-callout-color}{HTML}{909090}
\definecolor{quarto-callout-note-color}{HTML}{0758E5}
\definecolor{quarto-callout-important-color}{HTML}{CC1914}
\definecolor{quarto-callout-warning-color}{HTML}{EB9113}
\definecolor{quarto-callout-tip-color}{HTML}{00A047}
\definecolor{quarto-callout-caution-color}{HTML}{FC5300}
\definecolor{quarto-callout-color-frame}{HTML}{ACACAC}
\definecolor{quarto-callout-note-color-frame}{HTML}{4582EC}
\definecolor{quarto-callout-important-color-frame}{HTML}{D9534F}
\definecolor{quarto-callout-warning-color-frame}{HTML}{F0AD4E}
\definecolor{quarto-callout-tip-color-frame}{HTML}{02B875}
\definecolor{quarto-callout-caution-color-frame}{HTML}{FD7E14}

%\newlength\Oldarrayrulewidth
%\newlength\Oldtabcolsep


\usepackage{hyperref}




\providecommand{\tightlist}{%
  \setlength{\itemsep}{0pt}\setlength{\parskip}{0pt}}
\usepackage{longtable,booktabs,array}
\usepackage{calc} % for calculating minipage widths
% Correct order of tables after \paragraph or \subparagraph
\usepackage{etoolbox}
\makeatletter
\patchcmd\longtable{\par}{\if@noskipsec\mbox{}\fi\par}{}{}
\makeatother
% Allow footnotes in longtable head/foot
\IfFileExists{footnotehyper.sty}{\usepackage{footnotehyper}}{\usepackage{footnote}}
\makesavenoteenv{longtable}

\usepackage{graphicx}
\makeatletter
\newsavebox\pandoc@box
\newcommand*\pandocbounded[1]{% scales image to fit in text height/width
  \sbox\pandoc@box{#1}%
  \Gscale@div\@tempa{\textheight}{\dimexpr\ht\pandoc@box+\dp\pandoc@box\relax}%
  \Gscale@div\@tempb{\linewidth}{\wd\pandoc@box}%
  \ifdim\@tempb\p@<\@tempa\p@\let\@tempa\@tempb\fi% select the smaller of both
  \ifdim\@tempa\p@<\p@\scalebox{\@tempa}{\usebox\pandoc@box}%
  \else\usebox{\pandoc@box}%
  \fi%
}
% Set default figure placement to htbp
\def\fps@figure{htbp}
\makeatother







\usepackage{fontspec} 

\defaultfontfeatures{Scale=MatchLowercase}
\defaultfontfeatures[\rmfamily]{Ligatures=TeX,Scale=1}

  \setmainfont[,RawFeature={fallback=mainfontfallback}]{Latin Modern
Roman}




\title{Separando Hojas de Excel en documentos individuales: VBA}


\shorttitle{Combinar Hojas de Excel}


\usepackage{etoolbox}



\ccoppy{\textcopyright~2023}



\author{Edison Achalma}



\affiliation{
{Escuela Profesional de Economía, Universidad Nacional de San Cristóbal
de Huamanga}}




\leftheader{Achalma}

\date{2023-05-31}


\abstract{This article introduces a simple VBA macro designed to merge
all sheets from multiple Excel files into a single workbook. The macro
automates the process of data consolidation, which is particularly
useful for repetitive tasks in econometrics or any field requiring data
aggregation from various sources. The method involves opening each file
in read-only mode, copying all sheets to a target workbook, and then
closing the source workbook to proceed to the next file in the
directory. }

\keywords{VBA, Excel, Data Automation, Macros, Consolidation}

\authornote{\par{\addORCIDlink{Edison Achalma}{0000-0001-6996-3364}} 
\par{ }
\par{   El autor no tiene conflictos de interés que revelar.    Los
roles de autor se clasificaron utilizando la taxonomía de roles de
colaborador (CRediT; https://credit.niso.org/) de la siguiente
manera:  Edison Achalma:   conceptualización, redacción}
\par{La correspondencia relativa a este artículo debe dirigirse a Edison
Achalma, Email: \href{mailto:elmer.achalma.09@unsch.edu.pe}{elmer.achalma.09@unsch.edu.pe}}
}

\usepackage{pbalance}
% \usepackage{float}
\makeatletter
\let\oldtpt\ThreePartTable
\let\endoldtpt\endThreePartTable
\def\ThreePartTable{\@ifnextchar[\ThreePartTable@i \ThreePartTable@ii}
\def\ThreePartTable@i[#1]{\begin{figure}[!htbp]
\onecolumn
\begin{minipage}{0.485\textwidth}
\oldtpt[#1]
}
\def\ThreePartTable@ii{\begin{figure}[!htbp]
\onecolumn
\begin{minipage}{0.48\textwidth}
\oldtpt
}
\def\endThreePartTable{
\endoldtpt
\end{minipage}
\twocolumn
\end{figure}}
\makeatother


\makeatletter
\let\endoldlt\endlongtable		
\def\endlongtable{
\hline
\endoldlt}
\makeatother

\newenvironment{twocolumntable}% environment name
{% begin code
\begin{table*}[!htbp]%
\onecolumn%
}%
{%
\twocolumn%
\end{table*}%
}% end code

\urlstyle{same}



\makeatletter
\@ifpackageloaded{caption}{}{\usepackage{caption}}
\AtBeginDocument{%
\ifdefined\contentsname
  \renewcommand*\contentsname{Tabla de contenidos}
\else
  \newcommand\contentsname{Tabla de contenidos}
\fi
\ifdefined\listfigurename
  \renewcommand*\listfigurename{List of Figures}
\else
  \newcommand\listfigurename{List of Figures}
\fi
\ifdefined\listtablename
  \renewcommand*\listtablename{List of Tables}
\else
  \newcommand\listtablename{List of Tables}
\fi
\ifdefined\figurename
  \renewcommand*\figurename{Figura}
\else
  \newcommand\figurename{Figura}
\fi
\ifdefined\tablename
  \renewcommand*\tablename{Tabla}
\else
  \newcommand\tablename{Tabla}
\fi
}
\@ifpackageloaded{float}{}{\usepackage{float}}
\floatstyle{ruled}
\@ifundefined{c@chapter}{\newfloat{codelisting}{h}{lop}}{\newfloat{codelisting}{h}{lop}[chapter]}
\floatname{codelisting}{Listing}
\newcommand*\listoflistings{\listof{codelisting}{List of Listings}}
\makeatother
\makeatletter
\makeatother
\makeatletter
\@ifpackageloaded{caption}{}{\usepackage{caption}}
\@ifpackageloaded{subcaption}{}{\usepackage{subcaption}}
\makeatother
\makeatletter
\@ifpackageloaded{fontawesome5}{}{\usepackage{fontawesome5}}
\makeatother

% From https://tex.stackexchange.com/a/645996/211326
%%% apa7 doesn't want to add appendix section titles in the toc
%%% let's make it do it
\makeatletter
\xpatchcmd{\appendix}
  {\par}
  {\addcontentsline{toc}{section}{\@currentlabelname}\par}
  {}{}
\makeatother

%% Disable longtable counter
%% https://tex.stackexchange.com/a/248395/211326

\usepackage{etoolbox}

\makeatletter
\patchcmd{\LT@caption}
  {\bgroup}
  {\bgroup\global\LTpatch@captiontrue}
  {}{}
\patchcmd{\longtable}
  {\par}
  {\par\global\LTpatch@captionfalse}
  {}{}
\apptocmd{\endlongtable}
  {\ifLTpatch@caption\else\addtocounter{table}{-1}\fi}
  {}{}
\newif\ifLTpatch@caption
\makeatother

\begin{document}

\maketitle

\hypertarget{toc}{}
\tableofcontents
\newpage
\section[Introduction]{Separando Hojas de Excel en documentos
individuales}

\setcounter{secnumdepth}{5}

\setlength\LTleft{0pt}


SEPARAR HOJAS DE EXCEL EN DOCUMENTOS INDIVIDUALES Seguramente en
cualquier ocasion les han pedido que deben separar un documento en
excel, el cual contiene varias hojas, ya que por facilidad de
visualización y trabajo el documento se debe tener de esta manera, pero
al final debemos tener documentos individuales para ser enviados por
separado o simplemente para darle otro tratamiento. En este articulo
aprenderemos a separar ese documento el cual contiene varias hojas y
dejarlo en documentos individuales. Para ello solamente debemos ingresar
al modo desarrollador en excel y copiar unas lineas de codigo y ejecutar
y se realizara automaticamente la separacion. Se deben seguir los
siguientes pasos:

1- Creamos una carpeta donde vamos a guardar el documento el cual vamos
a separar. 2- Abrimos el documento y pulsamos las teclas
\texttt{alt+F11}. 3- De esta manera se abrirá el modo desarrollador, en
la parte izquierda encontramos un icono que dice ver código. damos clic
en el y se abrirá una ventana para insertar el código. Copiamos y
pegamos el siguiente fragmento de código:

\begin{verbatim}
Sub Splitbook()
'Updateby20140612
Dim xPath As String
xPath = Application.ActiveWorkbook.Path
Application.ScreenUpdating = False
Application.DisplayAlerts = False
For Each xWs In ThisWorkbook.Sheets
    xWs.Copy
    Application.ActiveWorkbook.SaveAs Filename:=xPath & "\" & xWs.Name & ".xlsx"
    Application.ActiveWorkbook.Close False
Next
Application.DisplayAlerts = True
Application.ScreenUpdating = True
End Sub
\end{verbatim}

4- Damos clic en el icono \textbf{Ejecutar Sub} el cual lo encontramos
en la barra superior con un icono triangular. 5- Inmediatamente se
ejecutará el código y vamos a visualizar la carpeta, encontrando las
hojas ya separadas en documentos individuales.

De esta manera se habrá solucionado esa tarea que tal vez puede resultar
tediosa si no conocemos mucho del modo desarrollador en excel.

\section{Publicaciones Similares}\label{publicaciones-similares}

Si te interesó este artículo, te recomendamos que explores otros blogs y
recursos relacionados que pueden ampliar tus conocimientos. Aquí te dejo
algunas sugerencias:

\begin{enumerate}
\def\labelenumi{\arabic{enumi}.}
\tightlist
\item
  \href{https://achalmaedison.netlify.app/herramientas-oficina/ofimatica/2022-12-05-01-introduccion-al-lenguaje-y-editor-vba/index.pdf}{\faIcon{file-pdf}}
  \href{https://achalmaedison.netlify.app/herramientas-oficina/ofimatica/2022-12-05-01-introduccion-al-lenguaje-y-editor-vba}{01
  Introduccion Al Lenguaje Y Editor Vba}
\item
  \href{https://achalmaedison.netlify.app/herramientas-oficina/ofimatica/2022-12-12-02-grabar-y-modificar/index.pdf}{\faIcon{file-pdf}}
  \href{https://achalmaedison.netlify.app/herramientas-oficina/ofimatica/2022-12-12-02-grabar-y-modificar}{02
  Grabar Y Modificar}
\item
  \href{https://achalmaedison.netlify.app/herramientas-oficina/ofimatica/2022-12-19-03-procedimientos/index.pdf}{\faIcon{file-pdf}}
  \href{https://achalmaedison.netlify.app/herramientas-oficina/ofimatica/2022-12-19-03-procedimientos}{03
  Procedimientos}
\item
  \href{https://achalmaedison.netlify.app/herramientas-oficina/ofimatica/2022-12-26-04-funciones-en-vba/index.pdf}{\faIcon{file-pdf}}
  \href{https://achalmaedison.netlify.app/herramientas-oficina/ofimatica/2022-12-26-04-funciones-en-vba}{04
  Funciones En Vba}
\item
  \href{https://achalmaedison.netlify.app/herramientas-oficina/ofimatica/2023-01-02-05-funciones-condicionales-estructuras-condicionales/index.pdf}{\faIcon{file-pdf}}
  \href{https://achalmaedison.netlify.app/herramientas-oficina/ofimatica/2023-01-02-05-funciones-condicionales-estructuras-condicionales}{05
  Funciones Condicionales Estructuras Condicionales}
\item
  \href{https://achalmaedison.netlify.app/herramientas-oficina/ofimatica/2023-01-09-06-funciones-iterativas-estructuras-repetitivas-o-bucles/index.pdf}{\faIcon{file-pdf}}
  \href{https://achalmaedison.netlify.app/herramientas-oficina/ofimatica/2023-01-09-06-funciones-iterativas-estructuras-repetitivas-o-bucles}{06
  Funciones Iterativas Estructuras Repetitivas O Bucles}
\item
  \href{https://achalmaedison.netlify.app/herramientas-oficina/ofimatica/2023-01-16-07-formularios/index.pdf}{\faIcon{file-pdf}}
  \href{https://achalmaedison.netlify.app/herramientas-oficina/ofimatica/2023-01-16-07-formularios}{07
  Formularios}
\item
  \href{https://achalmaedison.netlify.app/herramientas-oficina/ofimatica/2023-01-23-08-eventos/index.pdf}{\faIcon{file-pdf}}
  \href{https://achalmaedison.netlify.app/herramientas-oficina/ofimatica/2023-01-23-08-eventos}{08
  Eventos}
\item
  \href{https://achalmaedison.netlify.app/herramientas-oficina/ofimatica/2023-03-17-comando-para-convertir-docx-a-odt/index.pdf}{\faIcon{file-pdf}}
  \href{https://achalmaedison.netlify.app/herramientas-oficina/ofimatica/2023-03-17-comando-para-convertir-docx-a-odt}{Comando
  Para Convertir Docx A Odt}
\item
  \href{https://achalmaedison.netlify.app/herramientas-oficina/ofimatica/2023-04-03-buscar-reemplazar-en-libreoffice/index.pdf}{\faIcon{file-pdf}}
  \href{https://achalmaedison.netlify.app/herramientas-oficina/ofimatica/2023-04-03-buscar-reemplazar-en-libreoffice}{Buscar
  Reemplazar En Libreoffice}
\item
  \href{https://achalmaedison.netlify.app/herramientas-oficina/ofimatica/2023-05-21-anclaje-envoltura-alineacion-y-organizacion-de-objetos-en-llibreoffice/index.pdf}{\faIcon{file-pdf}}
  \href{https://achalmaedison.netlify.app/herramientas-oficina/ofimatica/2023-05-21-anclaje-envoltura-alineacion-y-organizacion-de-objetos-en-llibreoffice}{Anclaje
  Envoltura Alineacion Y Organizacion De Objetos En Llibreoffice}
\item
  \href{https://achalmaedison.netlify.app/herramientas-oficina/ofimatica/2023-05-31-combinando-hojas-de-excel-con-vba/index.pdf}{\faIcon{file-pdf}}
  \href{https://achalmaedison.netlify.app/herramientas-oficina/ofimatica/2023-05-31-combinando-hojas-de-excel-con-vba}{Combinando
  Hojas De Excel Con Vba}
\item
  \href{https://achalmaedison.netlify.app/herramientas-oficina/ofimatica/2023-06-05-separando-hojas-de-excel-en-documentos-individuales/index.pdf}{\faIcon{file-pdf}}
  \href{https://achalmaedison.netlify.app/herramientas-oficina/ofimatica/2023-06-05-separando-hojas-de-excel-en-documentos-individuales}{Separando
  Hojas De Excel En Documentos Individuales}
\item
  \href{https://achalmaedison.netlify.app/herramientas-oficina/ofimatica/2024-03-31-por-editar/index.pdf}{\faIcon{file-pdf}}
  \href{https://achalmaedison.netlify.app/herramientas-oficina/ofimatica/2024-03-31-por-editar}{Por
  Editar}
\end{enumerate}

Esperamos que encuentres estas publicaciones igualmente interesantes y
útiles. ¡Disfruta de la lectura!






\end{document}
