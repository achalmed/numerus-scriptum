\documentclass[
  jou,
  floatsintext,
  longtable,
  a4paper,
  nolmodern,
  notxfonts,
  notimes,
  colorlinks=true,linkcolor=blue,citecolor=blue,urlcolor=blue]{apa7}

\usepackage{amsmath}
\usepackage{amssymb}



\usepackage[bidi=default]{babel}
\babelprovide[main,import]{spanish}


\babelfont{rm}[,RawFeature={fallback=mainfontfallback}]{Latin Modern
Roman}
% get rid of language-specific shorthands (see #6817):
\let\LanguageShortHands\languageshorthands
\def\languageshorthands#1{}

\RequirePackage{longtable}
\RequirePackage{threeparttablex}

\makeatletter
\renewcommand{\paragraph}{\@startsection{paragraph}{4}{\parindent}%
	{0\baselineskip \@plus 0.2ex \@minus 0.2ex}%
	{-.5em}%
	{\normalfont\normalsize\bfseries\typesectitle}}

\renewcommand{\subparagraph}[1]{\@startsection{subparagraph}{5}{0.5em}%
	{0\baselineskip \@plus 0.2ex \@minus 0.2ex}%
	{-\z@\relax}%
	{\normalfont\normalsize\bfseries\itshape\hspace{\parindent}{#1}\textit{\addperi}}{\relax}}
\makeatother




\usepackage{longtable, booktabs, multirow, multicol, colortbl, hhline, caption, array, float, xpatch}
\usepackage{subcaption}


\renewcommand\thesubfigure{\Alph{subfigure}}
\setcounter{topnumber}{2}
\setcounter{bottomnumber}{2}
\setcounter{totalnumber}{4}
\renewcommand{\topfraction}{0.85}
\renewcommand{\bottomfraction}{0.85}
\renewcommand{\textfraction}{0.15}
\renewcommand{\floatpagefraction}{0.7}

\usepackage{tcolorbox}
\tcbuselibrary{listings,theorems, breakable, skins}
\usepackage{fontawesome5}

\definecolor{quarto-callout-color}{HTML}{909090}
\definecolor{quarto-callout-note-color}{HTML}{0758E5}
\definecolor{quarto-callout-important-color}{HTML}{CC1914}
\definecolor{quarto-callout-warning-color}{HTML}{EB9113}
\definecolor{quarto-callout-tip-color}{HTML}{00A047}
\definecolor{quarto-callout-caution-color}{HTML}{FC5300}
\definecolor{quarto-callout-color-frame}{HTML}{ACACAC}
\definecolor{quarto-callout-note-color-frame}{HTML}{4582EC}
\definecolor{quarto-callout-important-color-frame}{HTML}{D9534F}
\definecolor{quarto-callout-warning-color-frame}{HTML}{F0AD4E}
\definecolor{quarto-callout-tip-color-frame}{HTML}{02B875}
\definecolor{quarto-callout-caution-color-frame}{HTML}{FD7E14}

%\newlength\Oldarrayrulewidth
%\newlength\Oldtabcolsep


\usepackage{hyperref}



\usepackage{color}
\usepackage{fancyvrb}
\newcommand{\VerbBar}{|}
\newcommand{\VERB}{\Verb[commandchars=\\\{\}]}
\DefineVerbatimEnvironment{Highlighting}{Verbatim}{commandchars=\\\{\}}
% Add ',fontsize=\small' for more characters per line
\usepackage{framed}
\definecolor{shadecolor}{RGB}{241,243,245}
\newenvironment{Shaded}{\begin{snugshade}}{\end{snugshade}}
\newcommand{\AlertTok}[1]{\textcolor[rgb]{0.68,0.00,0.00}{#1}}
\newcommand{\AnnotationTok}[1]{\textcolor[rgb]{0.37,0.37,0.37}{#1}}
\newcommand{\AttributeTok}[1]{\textcolor[rgb]{0.40,0.45,0.13}{#1}}
\newcommand{\BaseNTok}[1]{\textcolor[rgb]{0.68,0.00,0.00}{#1}}
\newcommand{\BuiltInTok}[1]{\textcolor[rgb]{0.00,0.23,0.31}{#1}}
\newcommand{\CharTok}[1]{\textcolor[rgb]{0.13,0.47,0.30}{#1}}
\newcommand{\CommentTok}[1]{\textcolor[rgb]{0.37,0.37,0.37}{#1}}
\newcommand{\CommentVarTok}[1]{\textcolor[rgb]{0.37,0.37,0.37}{\textit{#1}}}
\newcommand{\ConstantTok}[1]{\textcolor[rgb]{0.56,0.35,0.01}{#1}}
\newcommand{\ControlFlowTok}[1]{\textcolor[rgb]{0.00,0.23,0.31}{\textbf{#1}}}
\newcommand{\DataTypeTok}[1]{\textcolor[rgb]{0.68,0.00,0.00}{#1}}
\newcommand{\DecValTok}[1]{\textcolor[rgb]{0.68,0.00,0.00}{#1}}
\newcommand{\DocumentationTok}[1]{\textcolor[rgb]{0.37,0.37,0.37}{\textit{#1}}}
\newcommand{\ErrorTok}[1]{\textcolor[rgb]{0.68,0.00,0.00}{#1}}
\newcommand{\ExtensionTok}[1]{\textcolor[rgb]{0.00,0.23,0.31}{#1}}
\newcommand{\FloatTok}[1]{\textcolor[rgb]{0.68,0.00,0.00}{#1}}
\newcommand{\FunctionTok}[1]{\textcolor[rgb]{0.28,0.35,0.67}{#1}}
\newcommand{\ImportTok}[1]{\textcolor[rgb]{0.00,0.46,0.62}{#1}}
\newcommand{\InformationTok}[1]{\textcolor[rgb]{0.37,0.37,0.37}{#1}}
\newcommand{\KeywordTok}[1]{\textcolor[rgb]{0.00,0.23,0.31}{\textbf{#1}}}
\newcommand{\NormalTok}[1]{\textcolor[rgb]{0.00,0.23,0.31}{#1}}
\newcommand{\OperatorTok}[1]{\textcolor[rgb]{0.37,0.37,0.37}{#1}}
\newcommand{\OtherTok}[1]{\textcolor[rgb]{0.00,0.23,0.31}{#1}}
\newcommand{\PreprocessorTok}[1]{\textcolor[rgb]{0.68,0.00,0.00}{#1}}
\newcommand{\RegionMarkerTok}[1]{\textcolor[rgb]{0.00,0.23,0.31}{#1}}
\newcommand{\SpecialCharTok}[1]{\textcolor[rgb]{0.37,0.37,0.37}{#1}}
\newcommand{\SpecialStringTok}[1]{\textcolor[rgb]{0.13,0.47,0.30}{#1}}
\newcommand{\StringTok}[1]{\textcolor[rgb]{0.13,0.47,0.30}{#1}}
\newcommand{\VariableTok}[1]{\textcolor[rgb]{0.07,0.07,0.07}{#1}}
\newcommand{\VerbatimStringTok}[1]{\textcolor[rgb]{0.13,0.47,0.30}{#1}}
\newcommand{\WarningTok}[1]{\textcolor[rgb]{0.37,0.37,0.37}{\textit{#1}}}

\providecommand{\tightlist}{%
  \setlength{\itemsep}{0pt}\setlength{\parskip}{0pt}}
\usepackage{longtable,booktabs,array}
\usepackage{calc} % for calculating minipage widths
% Correct order of tables after \paragraph or \subparagraph
\usepackage{etoolbox}
\makeatletter
\patchcmd\longtable{\par}{\if@noskipsec\mbox{}\fi\par}{}{}
\makeatother
% Allow footnotes in longtable head/foot
\IfFileExists{footnotehyper.sty}{\usepackage{footnotehyper}}{\usepackage{footnote}}
\makesavenoteenv{longtable}

\usepackage{graphicx}
\makeatletter
\newsavebox\pandoc@box
\newcommand*\pandocbounded[1]{% scales image to fit in text height/width
  \sbox\pandoc@box{#1}%
  \Gscale@div\@tempa{\textheight}{\dimexpr\ht\pandoc@box+\dp\pandoc@box\relax}%
  \Gscale@div\@tempb{\linewidth}{\wd\pandoc@box}%
  \ifdim\@tempb\p@<\@tempa\p@\let\@tempa\@tempb\fi% select the smaller of both
  \ifdim\@tempa\p@<\p@\scalebox{\@tempa}{\usebox\pandoc@box}%
  \else\usebox{\pandoc@box}%
  \fi%
}
% Set default figure placement to htbp
\def\fps@figure{htbp}
\makeatother







\usepackage{fontspec} 

\defaultfontfeatures{Scale=MatchLowercase}
\defaultfontfeatures[\rmfamily]{Ligatures=TeX,Scale=1}

  \setmainfont[,RawFeature={fallback=mainfontfallback}]{Latin Modern
Roman}




\title{Configuración de entornos virtuales con Anaconda: Aprende a
crear, activar, desactivar y eliminar entornos virtuales en Conda para
mantener tus proyectos organizados y libres de conflictos de
dependencias.}


\shorttitle{Editar}


\usepackage{etoolbox}



\ccoppy{\textcopyright~2025}



\author{Edison Achalma}



\affiliation{
{Escuela Profesional de Economía, Universidad Nacional de San Cristóbal
de Huamanga}}




\leftheader{Achalma}

\date{2020-06-20}


\abstract{Primer parrafo de abstrac }

\keywords{keyword1, keyword2}

\authornote{\par{\addORCIDlink{Edison Achalma}{0000-0001-6996-3364}} 
\par{ }
\par{   El autor no tiene conflictos de interés que revelar.    Los
roles de autor se clasificaron utilizando la taxonomía de roles de
colaborador (CRediT; https://credit.niso.org/) de la siguiente
manera:  Edison Achalma:   conceptualización, redacción}
\par{La correspondencia relativa a este artículo debe dirigirse a Edison
Achalma, Email: \href{mailto:elmer.achalma.09@unsch.edu.pe}{elmer.achalma.09@unsch.edu.pe}}
}

\usepackage{pbalance}
% \usepackage{float}
\makeatletter
\let\oldtpt\ThreePartTable
\let\endoldtpt\endThreePartTable
\def\ThreePartTable{\@ifnextchar[\ThreePartTable@i \ThreePartTable@ii}
\def\ThreePartTable@i[#1]{\begin{figure}[!htbp]
\onecolumn
\begin{minipage}{0.485\textwidth}
\oldtpt[#1]
}
\def\ThreePartTable@ii{\begin{figure}[!htbp]
\onecolumn
\begin{minipage}{0.48\textwidth}
\oldtpt
}
\def\endThreePartTable{
\endoldtpt
\end{minipage}
\twocolumn
\end{figure}}
\makeatother


\makeatletter
\let\endoldlt\endlongtable		
\def\endlongtable{
\hline
\endoldlt}
\makeatother

\newenvironment{twocolumntable}% environment name
{% begin code
\begin{table*}[!htbp]%
\onecolumn%
}%
{%
\twocolumn%
\end{table*}%
}% end code

\urlstyle{same}



\makeatletter
\@ifpackageloaded{caption}{}{\usepackage{caption}}
\AtBeginDocument{%
\ifdefined\contentsname
  \renewcommand*\contentsname{Tabla de contenidos}
\else
  \newcommand\contentsname{Tabla de contenidos}
\fi
\ifdefined\listfigurename
  \renewcommand*\listfigurename{List of Figures}
\else
  \newcommand\listfigurename{List of Figures}
\fi
\ifdefined\listtablename
  \renewcommand*\listtablename{List of Tables}
\else
  \newcommand\listtablename{List of Tables}
\fi
\ifdefined\figurename
  \renewcommand*\figurename{Figura}
\else
  \newcommand\figurename{Figura}
\fi
\ifdefined\tablename
  \renewcommand*\tablename{Tabla}
\else
  \newcommand\tablename{Tabla}
\fi
}
\@ifpackageloaded{float}{}{\usepackage{float}}
\floatstyle{ruled}
\@ifundefined{c@chapter}{\newfloat{codelisting}{h}{lop}}{\newfloat{codelisting}{h}{lop}[chapter]}
\floatname{codelisting}{Listing}
\newcommand*\listoflistings{\listof{codelisting}{List of Listings}}
\makeatother
\makeatletter
\makeatother
\makeatletter
\@ifpackageloaded{caption}{}{\usepackage{caption}}
\@ifpackageloaded{subcaption}{}{\usepackage{subcaption}}
\makeatother
\makeatletter
\@ifpackageloaded{fontawesome5}{}{\usepackage{fontawesome5}}
\makeatother

% From https://tex.stackexchange.com/a/645996/211326
%%% apa7 doesn't want to add appendix section titles in the toc
%%% let's make it do it
\makeatletter
\xpatchcmd{\appendix}
  {\par}
  {\addcontentsline{toc}{section}{\@currentlabelname}\par}
  {}{}
\makeatother

%% Disable longtable counter
%% https://tex.stackexchange.com/a/248395/211326

\usepackage{etoolbox}

\makeatletter
\patchcmd{\LT@caption}
  {\bgroup}
  {\bgroup\global\LTpatch@captiontrue}
  {}{}
\patchcmd{\longtable}
  {\par}
  {\par\global\LTpatch@captionfalse}
  {}{}
\apptocmd{\endlongtable}
  {\ifLTpatch@caption\else\addtocounter{table}{-1}\fi}
  {}{}
\newif\ifLTpatch@caption
\makeatother

\begin{document}

\maketitle

\hypertarget{toc}{}
\tableofcontents
\newpage
\section[Introduction]{Configuración de entornos virtuales con Anaconda}

\setcounter{secnumdepth}{5}

\setlength\LTleft{0pt}


\section{Introducción}\label{introducciuxf3n}

¡Bienvenidos al blog! En esta ocasión, exploraremos un tema fundamental
para cualquier desarrollador o entusiasta de Python: la configuración de
un entorno virtual utilizando Anaconda. Si alguna vez te has preguntado
cómo mantener tus proyectos de Python aislados, organizados y libres de
conflictos de dependencias, estás en el lugar correcto.

¿Alguna vez has encontrado problemas al trabajar en múltiples proyectos
de Python, donde las diferentes versiones de las bibliotecas y paquetes
interfieren entre sí? ¡No te preocupes! Configurar un entorno virtual es
la solución perfecta para mantener todo bajo control.

Imagina tener la capacidad de crear espacios de trabajo aislados y
personalizados para cada proyecto, sin preocuparte por conflictos entre
las versiones de las bibliotecas o paquetes. Con Anaconda, una poderosa
plataforma de gestión de paquetes y ambientes virtuales, puedes lograr
precisamente eso.

En este blog, te guiaremos paso a paso en la configuración de tu primer
entorno virtual utilizando Anaconda. Aprenderás cómo crear ambientes
virtuales, instalar paquetes y bibliotecas específicos, y alternar
fácilmente entre ellos para cada proyecto.

Ya sea que seas un desarrollador principiante o experimentado, este blog
te brindará los conocimientos necesarios para dominar el arte de la
configuración de entornos virtuales con Anaconda. ¡Prepárate para
simplificar tu vida como desarrollador de Python y llevar tus proyectos
al siguiente nivel!

¡Comencemos esta emocionante aventura de configuración de entornos
virtuales con Anaconda!

\section{¿Qué es un entorno
virtual?}\label{quuxe9-es-un-entorno-virtual}

Si estas en el mundo de la programación en Python, es posible que hayas
oído hablar del término ``entorno virtual''.

Cuando hablamos de un entorno virtual, nos referimos a un ambiente
aislado y autónomo donde puedes desarrollar y ejecutar tus proyectos de
Python de manera independiente. En otras palabras, es como tener una
burbuja protegiendo tus proyectos y asegurándote de que las bibliotecas
y paquetes que utilices no entren en conflicto con otras versiones o
dependencias de Python que puedas tener instaladas en tu sistema
operativo.

\subsection{¿Cuáles son los beneficios de trabajar con entornos
virtuales en proyectos de Python en
Linux?}\label{cuuxe1les-son-los-beneficios-de-trabajar-con-entornos-virtuales-en-proyectos-de-python-en-linux}

Permíteme destacar algunos puntos clave:

\begin{enumerate}
\def\labelenumi{\arabic{enumi}.}
\item
  \textbf{Aislamiento:} Los entornos virtuales te permiten tener control
  total sobre las versiones de las bibliotecas y paquetes que utilizas
  en tus proyectos. Esto significa que puedes crear un ambiente aislado
  para cada proyecto, evitando conflictos y problemas de compatibilidad
  entre diferentes versiones de Python y sus dependencias.
\item
  \textbf{Portabilidad:} Al utilizar entornos virtuales, puedes
  compartir fácilmente tus proyectos con otros desarrolladores o
  ejecutarlos en diferentes máquinas sin preocuparte por las diferencias
  en las configuraciones del sistema. Todo lo que necesitas es compartir
  el archivo de requisitos del entorno virtual y cualquiera podrá
  recrear el mismo ambiente de trabajo en su propia máquina.
\item
  \textbf{Mantenimiento sencillo:} Los entornos virtuales facilitan la
  gestión de tus proyectos. Puedes instalar y actualizar paquetes de
  forma independiente dentro de cada ambiente virtual, sin afectar a
  otros proyectos o al sistema operativo en sí. Además, si algo sale mal
  en un entorno virtual, puedes solucionarlo sin que afecte a tus otros
  proyectos.
\item
  \textbf{Experimentación segura:} Si quieres probar una nueva
  biblioteca o una versión diferente de una dependencia en particular,
  un entorno virtual te proporciona un espacio seguro para hacerlo.
  Puedes instalar y probar nuevas bibliotecas sin preocuparte de que
  afecten a otros proyectos o rompan la funcionalidad existente.
\end{enumerate}

\section{Anaconda}\label{anaconda}

¿Qué es Anaconda? Es mucho más que una simple herramienta, es una
plataforma completa de gestión de paquetes y entornos virtuales diseñada
específicamente para los amantes de Python como tú.

Entonces, ¿por qué deberías elegir Anaconda frente a otras herramientas?
Permíteme contarte algunas de las ventajas que hacen de Anaconda una
elección excepcional:

\begin{enumerate}
\def\labelenumi{\arabic{enumi}.}
\item
  \textbf{Gestión de paquetes simplificada:} Con Anaconda, olvídate de
  preocuparte por instalar y gestionar paquetes individualmente.
  Anaconda tiene su propio sistema de gestión de paquetes llamado
  ``conda'', que te permite instalar, actualizar y eliminar paquetes de
  manera sencilla y eficiente. Además, Anaconda cuenta con un amplio
  repositorio de paquetes precompilados que puedes explorar y utilizar
  en tus proyectos con un solo comando.
\item
  \textbf{Creación de entornos virtuales sin complicaciones:} ¿Recuerdas
  la importancia de los entornos virtuales que discutimos anteriormente?
  Bueno, con Anaconda, crear y gestionar entornos virtuales es pan
  comido. Puedes crear un entorno virtual aislado para cada proyecto en
  cuestión de segundos. Además, puedes especificar fácilmente la versión
  de Python y las dependencias requeridas para cada entorno virtual,
  manteniendo todo organizado y libre de conflictos.
\item
  \textbf{Multiplataforma:} Ya sea que estés utilizando Linux, Windows o
  macOS, Anaconda te tiene cubierto. Esta plataforma es compatible con
  múltiples sistemas operativos, lo que significa que puedes disfrutar
  de las mismas ventajas y características sin importar el entorno en el
  que te encuentres. Así que no importa si eres un entusiasta de Linux,
  un defensor de Windows o un fanático de macOS, Anaconda estará a tu
  lado.
\item
  \textbf{Integración con herramientas adicionales:} Anaconda no solo se
  detiene en la gestión de paquetes y entornos virtuales, sino que
  también ofrece una amplia gama de herramientas y utilidades
  adicionales que pueden mejorar tu flujo de trabajo. Desde el entorno
  de desarrollo integrado (IDE) llamado Anaconda Navigator, hasta la
  potencia de Jupyter Notebooks y la facilidad de distribución con
  Anaconda Cloud, hay muchas herramientas a tu disposición para mejorar
  tu productividad y simplificar tu desarrollo.
\end{enumerate}

\section{Instalación de Anaconda en Ubuntu
Linux}\label{instalaciuxf3n-de-anaconda-en-ubuntu-linux}

Si estás utilizando Ubuntu Linux y te emociona comenzar a trabajar con
Anaconda. Te guiaré a través de los pasos para descargar e instalar
Anaconda en tu sistema Ubuntu Linux.

\textbf{Paso 1: Descargar el instalador de Anaconda}

Para empezar, debes visitar el sitio web oficial de Anaconda
\href{https://www.anaconda.com/download/}{Descargar Anaconda} y
descargar el instalador adecuado para tu versión de Ubuntu Linux.
Asegúrate de seleccionar la versión de Python que deseas utilizar y si
tu sistema operativo es de 32 o 64 bits.

\textbf{Paso 2: Abrir la terminal}

Una vez que hayas descargado el instalador de Anaconda, abre la terminal
en tu sistema Ubuntu Linux. Puedes hacerlo utilizando el atajo de
teclado Ctrl + Alt + T o buscando ``Terminal'' en el menú de
aplicaciones.

\textbf{Paso 3: Navegar a la ubicación del instalador}

En la terminal, navega hasta la ubicación donde descargaste el
instalador de Anaconda. Por ejemplo, si lo descargaste en la carpeta
``Descargas'', puedes usar el comando siguiente para ir a esa ubicación:

\begin{Shaded}
\begin{Highlighting}[]
\BuiltInTok{cd}\NormalTok{ Descargas}
\end{Highlighting}
\end{Shaded}

Recuerda reemplazar ``Descargas'' con la carpeta en la que hayas
guardado el archivo.

\textbf{Paso 4: Ejecutar el instalador de Anaconda}

Una vez que estés en la ubicación del instalador de Anaconda, puedes
ejecutar el siguiente comando para iniciar el proceso de instalación:

\begin{Shaded}
\begin{Highlighting}[]
\FunctionTok{bash}\NormalTok{ nombre\_del\_instalador.sh}
\end{Highlighting}
\end{Shaded}

Asegúrate de reemplazar ``nombre\_del\_instalador.sh'' con el nombre
real del archivo que descargaste.

\textbf{Paso 5: Sigue las instrucciones de instalación}

Después de ejecutar el comando anterior, seguirás las instrucciones del
instalador de Anaconda en la terminal. Acepta los términos de licencia,
selecciona la ubicación de instalación y responde cualquier pregunta
adicional que se te presente durante el proceso de instalación.

\textbf{Paso 6: Añadir Anaconda al PATH del sistema}

Una vez que la instalación se complete, se te preguntará si deseas
agregar Anaconda al PATH del sistema. Es recomendable seleccionar
``yes'' para que puedas acceder a los comandos de Anaconda desde
cualquier ubicación en la terminal.

\section{Configuración de un entorno virtual en
Conda}\label{configuraciuxf3n-de-un-entorno-virtual-en-conda}

\textbf{Paso 1: Actualiza Conda}:

Antes de empezar, es una buena práctica asegurarte de tener la versión
más reciente de Conda. Abre tu terminal y ejecuta los siguientes
comandos:

\begin{Shaded}
\begin{Highlighting}[]
\ExtensionTok{conda}\NormalTok{ update conda}
\ExtensionTok{conda}\NormalTok{ update }\AttributeTok{{-}{-}all}
\end{Highlighting}
\end{Shaded}

Esto actualizará Conda y todos los paquetes asociados.

\textbf{Paso 2: Configura el entorno virtual:}

Ahora, vamos a crear un nuevo entorno virtual. En tu terminal, ejecuta
el siguiente comando:

\begin{Shaded}
\begin{Highlighting}[]
\ExtensionTok{conda}\NormalTok{ create }\AttributeTok{{-}{-}name}\NormalTok{ nombre\_del\_entorno}
\end{Highlighting}
\end{Shaded}

Reemplaza ``nombre\_del\_entorno'' con el nombre que desees darle a tu
entorno virtual.

\textbf{Paso 3: Activa el entorno virtual:}

Una vez que hayas creado el entorno virtual, puedes activarlo con el
siguiente comando:

\begin{Shaded}
\begin{Highlighting}[]
\ExtensionTok{conda}\NormalTok{ activate nombre\_del\_entorno}
\end{Highlighting}
\end{Shaded}

Esto te permitirá trabajar en el entorno virtual específico.

\textbf{Paso 4: Desactiva el entorno virtual:}

Cuando hayas terminado de trabajar en tu entorno virtual y desees volver
al entorno base, simplemente ejecuta el siguiente comando:

\begin{Shaded}
\begin{Highlighting}[]
\ExtensionTok{conda}\NormalTok{ deactivate}
\end{Highlighting}
\end{Shaded}

Esto te llevará de vuelta al entorno base de Conda.

\textbf{Paso 5: Elimina el entorno virtual:}

Si en algún momento deseas eliminar un entorno virtual, ejecuta el
siguiente comando:

\begin{Shaded}
\begin{Highlighting}[]
\ExtensionTok{conda}\NormalTok{ env remove }\AttributeTok{{-}{-}name}\NormalTok{ nombre\_del\_entorno}
\end{Highlighting}
\end{Shaded}

Asegúrate de reemplazar ``nombre\_del\_entorno'' con el nombre real del
entorno que deseas eliminar.

\textbf{Paso 6: Cambiar entorno}

\begin{enumerate}
\def\labelenumi{\arabic{enumi}.}
\item
  Abre Anaconda Navigator o el Anaconda Prompt (puedes encontrarlo en el
  menú de inicio de tu sistema operativo).
\item
  Una vez que hayas abierto el entorno de Anaconda, puedes verificar los
  entornos disponibles ejecutando el siguiente comando en el Anaconda
  Prompt:

\begin{Shaded}
\begin{Highlighting}[]
\ExtensionTok{conda}\NormalTok{ env list}
\end{Highlighting}
\end{Shaded}

  Esto te mostrará una lista de todos los entornos existentes.
\item
  Para cambiar al entorno de aprendizaje (llamado ``learn'' en este
  caso), utiliza el siguiente comando:

\begin{Shaded}
\begin{Highlighting}[]
\ExtensionTok{conda}\NormalTok{ activate learn}
\end{Highlighting}
\end{Shaded}

  Esto activará el entorno de aprendizaje y te permitirá trabajar en él.
\item
  Una vez en el entorno de aprendizaje, es posible que notes que no
  tiene instalados otros paquetes, aparte de los paquetes oficiales que
  vienen con Python. Si deseas tener un entorno relativamente limpio,
  puedes seguir estos pasos:

  \begin{itemize}
  \item
    Ejecuta el siguiente comando para abrir el intérprete de Python:

\begin{Shaded}
\begin{Highlighting}[]
\ExtensionTok{python}
\end{Highlighting}
\end{Shaded}
  \item
    Una vez dentro del intérprete de Python, ingresa el siguiente
    comando para importar el paquete ``requests'':

\begin{Shaded}
\begin{Highlighting}[]
\ImportTok{import}\NormalTok{ requests}
\end{Highlighting}
\end{Shaded}

    Verás que se muestra un mensaje indicando que no se puede encontrar
    el paquete ``requests'', lo cual es normal.
  \item
    Para salir del intérprete de Python, simplemente ingresa el
    siguiente comando:

\begin{Shaded}
\begin{Highlighting}[]
\NormalTok{exit()}
\end{Highlighting}
\end{Shaded}

    Con esto, saldrás del intérprete de Python y volverás al Anaconda
    Prompt.
  \end{itemize}
\end{enumerate}

¡Y eso es todo! Ahora tienes los pasos detallados para configurar y
administrar entornos virtuales en Conda.

\section{Instalación de paquetes y bibliotecas en un entorno
virtual}\label{instalaciuxf3n-de-paquetes-y-bibliotecas-en-un-entorno-virtual}

Cuando trabajas en proyectos de Python, es esencial tener acceso a las
herramientas y funcionalidades adecuadas.

\subsection{Uso de conda install}\label{uso-de-conda-install}

\begin{enumerate}
\def\labelenumi{\arabic{enumi}.}
\item
  Asegúrate de tener tu entorno virtual activado. Si aún no lo has
  hecho, consulta el artículo anterior para aprender cómo activar tu
  entorno virtual específico.
\item
  Abre tu terminal o línea de comandos y ejecuta el siguiente comando
  para instalar un paquete desde el repositorio de Anaconda:

\begin{Shaded}
\begin{Highlighting}[]
\ExtensionTok{conda}\NormalTok{ install nombre\_del\_paquete}
\end{Highlighting}
\end{Shaded}

  Asegúrate de reemplazar ``nombre\_del\_paquete'' con el nombre real
  del paquete que deseas instalar.
\item
  Conda buscará el paquete en el repositorio de Anaconda y gestionará
  las dependencias automáticamente. Sigue las instrucciones en la
  terminal para confirmar la instalación.
\end{enumerate}

\subsection{Uso de pip install}\label{uso-de-pip-install}

\begin{enumerate}
\def\labelenumi{\arabic{enumi}.}
\item
  Al igual que antes, asegúrate de tener tu entorno virtual activado.
\item
  Ejecuta el siguiente comando en tu terminal para instalar un paquete
  desde el Python Package Index (PyPI):

\begin{Shaded}
\begin{Highlighting}[]
\ExtensionTok{pip}\NormalTok{ install nombre\_del\_paquete}
\end{Highlighting}
\end{Shaded}

  Asegúrate de reemplazar ``nombre\_del\_paquete'' con el nombre real
  del paquete que deseas instalar.
\item
  Pip descargará el paquete desde PyPI y lo instalará en tu entorno
  virtual. Si el paquete tiene dependencias, pip también se encargará de
  resolverlas.
\end{enumerate}

Conda es especialmente útil para instalar paquetes que son parte del
repositorio de Anaconda, mientras que pip es más adecuado para paquetes
que se encuentran en PyPI. Ambas herramientas son poderosas y te
permiten acceder a una amplia gama de paquetes y bibliotecas para tus
proyectos.

\subsection{Ver información del paquete de
entorno:}\label{ver-informaciuxf3n-del-paquete-de-entorno}

Para ver todos los paquetes instalados en el entorno actual, puedes
utilizar el siguiente comando:

\begin{Shaded}
\begin{Highlighting}[]
\ExtensionTok{conda}\NormalTok{ list}
\end{Highlighting}
\end{Shaded}

Al ejecutar este comando en el Anaconda Prompt, se mostrará una lista de
todos los paquetes instalados en el entorno activo. Esto te permitirá
conocer los paquetes y sus respectivas versiones que están disponibles
en ese entorno.

\subsection{Importar y exportar
entornos:}\label{importar-y-exportar-entornos}

Si deseas exportar la información del paquete del entorno actual, puedes
utilizar el siguiente comando:

\begin{Shaded}
\begin{Highlighting}[]
\ExtensionTok{conda}\NormalTok{ env export }\OperatorTok{\textgreater{}}\NormalTok{ environment.yaml}
\end{Highlighting}
\end{Shaded}

Este comando guarda la información del paquete en un archivo YAML
llamado ``environment.yaml''. El archivo contendrá la lista de paquetes
y sus versiones que están instalados en el entorno actual.

Esta funcionalidad es útil cuando necesitas recrear el mismo entorno
virtual en otro lugar. Para crear un nuevo entorno virtual utilizando el
archivo de configuración, puedes utilizar el siguiente comando:

\begin{Shaded}
\begin{Highlighting}[]
\ExtensionTok{conda}\NormalTok{ env create }\AttributeTok{{-}f}\NormalTok{ environment.yaml}
\end{Highlighting}
\end{Shaded}

Este comando creará un nuevo entorno virtual utilizando el archivo de
configuración ``environment.yaml''. El nuevo entorno tendrá los mismos
paquetes y versiones que el entorno original, lo que facilita la
replicación del mismo entorno en diferentes sistemas.

Estos pasos son útiles para compartir y recrear entornos virtuales con
la misma configuración, lo que asegura que todos los paquetes necesarios
estén disponibles.

\section{Coclusión}\label{coclusiuxf3n}

En conclusión, el uso de Anaconda se presenta como una solución elegante
y sencilla para abordar las desventajas de entorno de Python. A través
de Anaconda, se puede gestionar de manera eficiente la instalación y
actualización de paquetes, así como la creación y exportación de
entornos virtuales. Sin embargo, es importante destacar que la
implementación de estas funcionalidades no es mágica, requiere
comprensión y familiaridad con los comandos y procesos asociados.

Además de la gestión de paquetes, Anaconda ofrece una amplia gama de
herramientas y paquetes para el análisis de datos, lo cual constituye
otro aspecto valioso de su funcionalidad. Sin embargo, en este contexto,
nos hemos enfocado en aprender cómo utilizar Anaconda para cambiar el
entorno de desarrollo de manera efectiva, lo cual ha representado una
mejora significativa en comparación con el enfoque tradicional.

\textbf{¡Happy coding!}

\section{Publicaciones Similares}\label{publicaciones-similares}

Si te interesó este artículo, te recomendamos que explores otros blogs y
recursos relacionados que pueden ampliar tus conocimientos. Aquí te dejo
algunas sugerencias:

\begin{enumerate}
\def\labelenumi{\arabic{enumi}.}
\tightlist
\item
  \href{https://achalmaedison.netlify.app/programacion-software/python/2020-06-19-instalacion-de-anaconda/index.pdf}{\faIcon{file-pdf}}
  \href{https://achalmaedison.netlify.app/programacion-software/python/2020-06-19-instalacion-de-anaconda}{Instalacion
  De Anaconda}
\item
  \href{https://achalmaedison.netlify.app/programacion-software/python/2020-06-20-configurar-entorno-virtual-python-anaconda/index.pdf}{\faIcon{file-pdf}}
  \href{https://achalmaedison.netlify.app/programacion-software/python/2020-06-20-configurar-entorno-virtual-python-anaconda}{Configurar
  Entorno Virtual Python Anaconda}
\item
  \href{https://achalmaedison.netlify.app/programacion-software/python/2021-04-17-01-introducion-a-la-programacion-con-python/index.pdf}{\faIcon{file-pdf}}
  \href{https://achalmaedison.netlify.app/programacion-software/python/2021-04-17-01-introducion-a-la-programacion-con-python}{01
  Introducion A La Programacion Con Python}
\item
  \href{https://achalmaedison.netlify.app/programacion-software/python/2021-05-31-02-variables-expresiones-y-statements-con-python/index.pdf}{\faIcon{file-pdf}}
  \href{https://achalmaedison.netlify.app/programacion-software/python/2021-05-31-02-variables-expresiones-y-statements-con-python}{02
  Variables Expresiones Y Statements Con Python}
\item
  \href{https://achalmaedison.netlify.app/programacion-software/python/2021-06-07-03-objetos-de-python/index.pdf}{\faIcon{file-pdf}}
  \href{https://achalmaedison.netlify.app/programacion-software/python/2021-06-07-03-objetos-de-python}{03
  Objetos De Python}
\item
  \href{https://achalmaedison.netlify.app/programacion-software/python/2021-06-14-04-ejecucion-condicional-con-python/index.pdf}{\faIcon{file-pdf}}
  \href{https://achalmaedison.netlify.app/programacion-software/python/2021-06-14-04-ejecucion-condicional-con-python}{04
  Ejecucion Condicional Con Python}
\item
  \href{https://achalmaedison.netlify.app/programacion-software/python/2021-06-21-05-iteraciones-con-python/index.pdf}{\faIcon{file-pdf}}
  \href{https://achalmaedison.netlify.app/programacion-software/python/2021-06-21-05-iteraciones-con-python}{05
  Iteraciones Con Python}
\item
  \href{https://achalmaedison.netlify.app/programacion-software/python/2021-08-16-06-funciones-con-python/index.pdf}{\faIcon{file-pdf}}
  \href{https://achalmaedison.netlify.app/programacion-software/python/2021-08-16-06-funciones-con-python}{06
  Funciones Con Python}
\item
  \href{https://achalmaedison.netlify.app/programacion-software/python/2021-08-23-07-dataframes-con-python/index.pdf}{\faIcon{file-pdf}}
  \href{https://achalmaedison.netlify.app/programacion-software/python/2021-08-23-07-dataframes-con-python}{07
  Dataframes Con Python}
\item
  \href{https://achalmaedison.netlify.app/programacion-software/python/2021-11-29-08-prediccion-y-metrica-de-performance-con-python/index.pdf}{\faIcon{file-pdf}}
  \href{https://achalmaedison.netlify.app/programacion-software/python/2021-11-29-08-prediccion-y-metrica-de-performance-con-python}{08
  Prediccion Y Metrica De Performance Con Python}
\item
  \href{https://achalmaedison.netlify.app/programacion-software/python/2021-12-06-09-metodos-de-machine-learning-para-clasificacion-con-python/index.pdf}{\faIcon{file-pdf}}
  \href{https://achalmaedison.netlify.app/programacion-software/python/2021-12-06-09-metodos-de-machine-learning-para-clasificacion-con-python}{09
  Metodos De Machine Learning Para Clasificacion Con Python}
\item
  \href{https://achalmaedison.netlify.app/programacion-software/python/2021-12-13-10-metodos-de-machine-learning-para-regresion-con-python/index.pdf}{\faIcon{file-pdf}}
  \href{https://achalmaedison.netlify.app/programacion-software/python/2021-12-13-10-metodos-de-machine-learning-para-regresion-con-python}{10
  Metodos De Machine Learning Para Regresion Con Python}
\item
  \href{https://achalmaedison.netlify.app/programacion-software/python/2022-10-31-11-validacion-cruzada-y-composicion-del-modelo-con-python/index.pdf}{\faIcon{file-pdf}}
  \href{https://achalmaedison.netlify.app/programacion-software/python/2022-10-31-11-validacion-cruzada-y-composicion-del-modelo-con-python}{11
  Validacion Cruzada Y Composicion Del Modelo Con Python}
\item
  \href{https://achalmaedison.netlify.app/programacion-software/python/2025-05-10-visualizacion-de-datos-con-python/index.pdf}{\faIcon{file-pdf}}
  \href{https://achalmaedison.netlify.app/programacion-software/python/2025-05-10-visualizacion-de-datos-con-python}{Visualizacion
  De Datos Con Python}
\end{enumerate}

Esperamos que encuentres estas publicaciones igualmente interesantes y
útiles. ¡Disfruta de la lectura!






\end{document}
