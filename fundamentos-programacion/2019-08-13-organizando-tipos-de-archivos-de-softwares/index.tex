\documentclass[
  stu,
  floatsintext,
  longtable,
  a4paper,
  nolmodern,
  notxfonts,
  notimes,
  colorlinks=true,linkcolor=blue,citecolor=blue,urlcolor=blue]{apa7}

\usepackage{amsmath}
\usepackage{amssymb}



\usepackage[bidi=default]{babel}
\babelprovide[main,import]{spanish}


\babelfont{rm}[,RawFeature={fallback=mainfontfallback}]{Latin Modern
Roman}
% get rid of language-specific shorthands (see #6817):
\let\LanguageShortHands\languageshorthands
\def\languageshorthands#1{}

\RequirePackage{longtable}
\RequirePackage{threeparttablex}

\makeatletter
\renewcommand{\paragraph}{\@startsection{paragraph}{4}{\parindent}%
	{0\baselineskip \@plus 0.2ex \@minus 0.2ex}%
	{-.5em}%
	{\normalfont\normalsize\bfseries\typesectitle}}

\renewcommand{\subparagraph}[1]{\@startsection{subparagraph}{5}{0.5em}%
	{0\baselineskip \@plus 0.2ex \@minus 0.2ex}%
	{-\z@\relax}%
	{\normalfont\normalsize\bfseries\itshape\hspace{\parindent}{#1}\textit{\addperi}}{\relax}}
\makeatother




\usepackage{longtable, booktabs, multirow, multicol, colortbl, hhline, caption, array, float, xpatch}
\usepackage{subcaption}


\renewcommand\thesubfigure{\Alph{subfigure}}
\setcounter{topnumber}{2}
\setcounter{bottomnumber}{2}
\setcounter{totalnumber}{4}
\renewcommand{\topfraction}{0.85}
\renewcommand{\bottomfraction}{0.85}
\renewcommand{\textfraction}{0.15}
\renewcommand{\floatpagefraction}{0.7}

\usepackage{tcolorbox}
\tcbuselibrary{listings,theorems, breakable, skins}
\usepackage{fontawesome5}

\definecolor{quarto-callout-color}{HTML}{909090}
\definecolor{quarto-callout-note-color}{HTML}{0758E5}
\definecolor{quarto-callout-important-color}{HTML}{CC1914}
\definecolor{quarto-callout-warning-color}{HTML}{EB9113}
\definecolor{quarto-callout-tip-color}{HTML}{00A047}
\definecolor{quarto-callout-caution-color}{HTML}{FC5300}
\definecolor{quarto-callout-color-frame}{HTML}{ACACAC}
\definecolor{quarto-callout-note-color-frame}{HTML}{4582EC}
\definecolor{quarto-callout-important-color-frame}{HTML}{D9534F}
\definecolor{quarto-callout-warning-color-frame}{HTML}{F0AD4E}
\definecolor{quarto-callout-tip-color-frame}{HTML}{02B875}
\definecolor{quarto-callout-caution-color-frame}{HTML}{FD7E14}

%\newlength\Oldarrayrulewidth
%\newlength\Oldtabcolsep


\usepackage{hyperref}




\providecommand{\tightlist}{%
  \setlength{\itemsep}{0pt}\setlength{\parskip}{0pt}}
\usepackage{longtable,booktabs,array}
\usepackage{calc} % for calculating minipage widths
% Correct order of tables after \paragraph or \subparagraph
\usepackage{etoolbox}
\makeatletter
\patchcmd\longtable{\par}{\if@noskipsec\mbox{}\fi\par}{}{}
\makeatother
% Allow footnotes in longtable head/foot
\IfFileExists{footnotehyper.sty}{\usepackage{footnotehyper}}{\usepackage{footnote}}
\makesavenoteenv{longtable}

\usepackage{graphicx}
\makeatletter
\newsavebox\pandoc@box
\newcommand*\pandocbounded[1]{% scales image to fit in text height/width
  \sbox\pandoc@box{#1}%
  \Gscale@div\@tempa{\textheight}{\dimexpr\ht\pandoc@box+\dp\pandoc@box\relax}%
  \Gscale@div\@tempb{\linewidth}{\wd\pandoc@box}%
  \ifdim\@tempb\p@<\@tempa\p@\let\@tempa\@tempb\fi% select the smaller of both
  \ifdim\@tempa\p@<\p@\scalebox{\@tempa}{\usebox\pandoc@box}%
  \else\usebox{\pandoc@box}%
  \fi%
}
% Set default figure placement to htbp
\def\fps@figure{htbp}
\makeatother







\usepackage{fontspec} 

\defaultfontfeatures{Scale=MatchLowercase}
\defaultfontfeatures[\rmfamily]{Ligatures=TeX,Scale=1}

  \setmainfont[,RawFeature={fallback=mainfontfallback}]{Latin Modern
Roman}




\title{Tipografía en entornos académicos y de programación: Selección y
aplicación para mejorar la legibilidad}


\shorttitle{Tipografía en academia y programación}


\usepackage{etoolbox}



\ccoppy{\textcopyright~2019}



\author{Edison Achalma}



\affiliation{
{Escuela Profesional de Economía, Universidad Nacional de San Cristóbal
de Huamanga}}




\leftheader{Achalma}

\date{2025-01-12}


\abstract{Typography is a fundamental element in both academic and
programming environments, impacting readability, comprehension, and
professional presentation. This article explores the principles of
typographic selection and application, emphasizing font choices for
academic writing, coding environments, and digital interfaces. It
discusses the characteristics of monospaced fonts for programming, the
importance of typographic contrast, and the role of ligatures in code
readability. Additionally, it provides insights into the differences
between OTF and TTF formats, along with recommendations for font pairing
in scholarly and technical contexts. Understanding typography enhances
clarity and effectiveness in both educational and software development
fields. }

\keywords{Typography, academic writing, programming
fonts, readability, monospaced typefaces}

\authornote{\par{\addORCIDlink{Edison Achalma}{0000-0001-6996-3364}} 
\par{ }
\par{   El autor no tiene conflictos de interés que revelar.    Los
roles de autor se clasificaron utilizando la taxonomía de roles de
colaborador (CRediT; https://credit.niso.org/) de la siguiente
manera:  Edison Achalma:   conceptualización, redacción}
\par{La correspondencia relativa a este artículo debe dirigirse a Edison
Achalma, Email: \href{mailto:elmer.achalma.09@unsch.edu.pe}{elmer.achalma.09@unsch.edu.pe}}
}

\makeatletter
\let\endoldlt\endlongtable
\def\endlongtable{
\hline
\endoldlt
}
\makeatother

\urlstyle{same}



\makeatletter
\@ifpackageloaded{caption}{}{\usepackage{caption}}
\AtBeginDocument{%
\ifdefined\contentsname
  \renewcommand*\contentsname{Tabla de contenidos}
\else
  \newcommand\contentsname{Tabla de contenidos}
\fi
\ifdefined\listfigurename
  \renewcommand*\listfigurename{List of Figures}
\else
  \newcommand\listfigurename{List of Figures}
\fi
\ifdefined\listtablename
  \renewcommand*\listtablename{List of Tables}
\else
  \newcommand\listtablename{List of Tables}
\fi
\ifdefined\figurename
  \renewcommand*\figurename{Figura}
\else
  \newcommand\figurename{Figura}
\fi
\ifdefined\tablename
  \renewcommand*\tablename{Tabla}
\else
  \newcommand\tablename{Tabla}
\fi
}
\@ifpackageloaded{float}{}{\usepackage{float}}
\floatstyle{ruled}
\@ifundefined{c@chapter}{\newfloat{codelisting}{h}{lop}}{\newfloat{codelisting}{h}{lop}[chapter]}
\floatname{codelisting}{Listing}
\newcommand*\listoflistings{\listof{codelisting}{List of Listings}}
\makeatother
\makeatletter
\makeatother
\makeatletter
\@ifpackageloaded{caption}{}{\usepackage{caption}}
\@ifpackageloaded{subcaption}{}{\usepackage{subcaption}}
\makeatother
\makeatletter
\@ifpackageloaded{fontawesome5}{}{\usepackage{fontawesome5}}
\makeatother

% From https://tex.stackexchange.com/a/645996/211326
%%% apa7 doesn't want to add appendix section titles in the toc
%%% let's make it do it
\makeatletter
\xpatchcmd{\appendix}
  {\par}
  {\addcontentsline{toc}{section}{\@currentlabelname}\par}
  {}{}
\makeatother

%% Disable longtable counter
%% https://tex.stackexchange.com/a/248395/211326

\usepackage{etoolbox}

\makeatletter
\patchcmd{\LT@caption}
  {\bgroup}
  {\bgroup\global\LTpatch@captiontrue}
  {}{}
\patchcmd{\longtable}
  {\par}
  {\par\global\LTpatch@captionfalse}
  {}{}
\apptocmd{\endlongtable}
  {\ifLTpatch@caption\else\addtocounter{table}{-1}\fi}
  {}{}
\newif\ifLTpatch@caption
\makeatother

\begin{document}

\maketitle

\hypertarget{toc}{}
\tableofcontents
\newpage
\section[Introduction]{Tipografía en entornos académicos y de
programación}

\setcounter{secnumdepth}{5}

\setlength\LTleft{0pt}


¡Absolutamente! Aquí tienes una tabla comparativa de los tipos de
archivos más comunes que generan los software estadísticos y de
programación que mencionaste, junto con sus funciones principales y
extensiones comunes. Esta tabla te será de gran utilidad para organizar
tus archivos de manera eficiente:

\begin{longtable}[]{@{}
  >{\raggedright\arraybackslash}p{(\linewidth - 6\tabcolsep) * \real{0.0965}}
  >{\raggedright\arraybackslash}p{(\linewidth - 6\tabcolsep) * \real{0.1316}}
  >{\raggedright\arraybackslash}p{(\linewidth - 6\tabcolsep) * \real{0.5789}}
  >{\raggedright\arraybackslash}p{(\linewidth - 6\tabcolsep) * \real{0.1930}}@{}}
\toprule\noalign{}
\begin{minipage}[b]{\linewidth}\raggedright
Software
\end{minipage} & \begin{minipage}[b]{\linewidth}\raggedright
Tipo de Archivo
\end{minipage} & \begin{minipage}[b]{\linewidth}\raggedright
Función Principal
\end{minipage} & \begin{minipage}[b]{\linewidth}\raggedright
Extensión Común
\end{minipage} \\
\midrule\noalign{}
\endhead
\bottomrule\noalign{}
\endlastfoot
\textbf{SPSS} & Datos & Almacena datos para análisis & .sav \\
& Sintaxis & Contiene comandos para ejecutar análisis & .sps \\
& Salida & Guarda resultados de análisis & .spv \\
& Gráficos & Almacena gráficos generados & .png, .jpg, .bmp \\
\textbf{Minitab} & Proyecto & Contiene todo el proyecto, incluyendo
datos, análisis y resultados & .mpj \\
& Hoja de Trabajo & Almacena datos para un análisis específico & .mtw \\
& Sesión & Registra la sesión de trabajo & .mtw \\
& Gráficos & Almacena gráficos generados & .png, .jpg, .bmp \\
& Macro & Contiene comandos para automatizar tareas & .mac \\
\textbf{EViews} & Workfile & Contiene todos los objetos creados en una
sesión & .wf1 \\
& Programa & Contiene código para automatizar tareas & .prg \\
& Datos & Almacena datos en formato ASCII & .dat \\
& Serie & Almacena una serie de tiempo & .ser \\
& Ecuación & Almacena una ecuación estimada & .equ \\
& Sistema & Almacena un sistema de ecuaciones & .sys \\
& Gráfico & Almacena gráficos generados & .wpg \\
\textbf{Gretl} & Datos & Almacena datos de series de tiempo o de corte
transversal & .gdt \\
& Resultados & Guarda resultados de estimaciones & .gph \\
& Gráficos & Almacena gráficos generados & .png, .jpg, .pdf \\
& Script & Contiene comandos de gretl & .gpi \\
\textbf{Stata} & Datos & Almacena datos para análisis & .dta \\
& Do-file & Contiene comandos de Stata & .do \\
& Log & Registra la ejecución de comandos & .log \\
& Resultados & Almacena resultados de comandos & .smcl \\
& Gráficos & Almacena gráficos generados & .gph \\
\textbf{Octave} & Datos & Matrices y vectores & .mat \\
& Script & Contiene código de Octave & .m \\
& Gráficos & Almacena gráficos generados & .png, .jpg, .pdf \\
\textbf{MATLAB} & Datos & Matrices y vectores & .mat \\
& Script & Contiene código de MATLAB & .m \\
& Gráficos & Almacena gráficos generados & .fig, .png, .jpg, .pdf \\
\textbf{Julia} & Datos & Matrices y vectores & .jld2 \\
& Script & Contiene código de Julia & .jl \\
& Gráficos & Almacena gráficos generados & .png, .jpg, .pdf \\
\textbf{R} & Datos & Data frames & .RData \\
& Script & Contiene código de R & .R \\
& Gráficos & Almacena gráficos generados & .png, .jpg, .pdf \\
\end{longtable}

\textbf{Observaciones:}

\begin{itemize}
\tightlist
\item
  \textbf{Extensiones comunes:} Aunque las extensiones pueden variar
  ligeramente entre diferentes versiones o configuraciones, las
  mencionadas son las más comunes.
\item
  \textbf{Formatos de datos:} Muchos software pueden importar y exportar
  datos en diversos formatos (CSV, Excel, etc.), pero sus formatos
  nativos son los más eficientes.
\item
  \textbf{Scripts:} Los scripts son fundamentales para automatizar
  tareas y reproducir análisis.
\item
  \textbf{Gráficos:} Los gráficos suelen guardarse en formatos de imagen
  estándar para su inclusión en informes.
\end{itemize}

\textbf{Recomendaciones adicionales:}

\begin{itemize}
\tightlist
\item
  \textbf{Organiza tus archivos:} Crea una estructura de carpetas clara
  y lógica para cada proyecto.
\item
  \textbf{Utiliza nombres descriptivos:} Facilita la identificación de
  los archivos.
\item
  \textbf{Realiza copias de seguridad:} Evita la pérdida de datos
  importantes.
\item
  \textbf{Documenta tu trabajo:} Añade comentarios a tus scripts para
  explicar el código.
\end{itemize}

\textbf{¿Necesitas más información sobre algún software o tipo de
archivo en particular?}

¡Absolutamente! Aquí tienes una matriz que agrupa las extensiones de
archivo más comunes utilizadas en software estadístico, según su función
principal. Ten en cuenta que algunas extensiones pueden tener usos más
específicos o variar ligeramente entre diferentes versiones de software:

\begin{longtable}[]{@{}
  >{\raggedright\arraybackslash}p{(\linewidth - 4\tabcolsep) * \real{0.2414}}
  >{\raggedright\arraybackslash}p{(\linewidth - 4\tabcolsep) * \real{0.3017}}
  >{\raggedright\arraybackslash}p{(\linewidth - 4\tabcolsep) * \real{0.4569}}@{}}
\toprule\noalign{}
\begin{minipage}[b]{\linewidth}\raggedright
Función Principal
\end{minipage} & \begin{minipage}[b]{\linewidth}\raggedright
Extensiones Comunes
\end{minipage} & \begin{minipage}[b]{\linewidth}\raggedright
Software
\end{minipage} \\
\midrule\noalign{}
\endhead
\bottomrule\noalign{}
\endlastfoot
\textbf{Datos} & .sav, .dta, .gdt, .mtw, .csv, .xlsx & SPSS, Stata,
Gretl, Minitab, Excel \\
\textbf{Sintaxis/Script} & .sps, .do, .gpi, .mac, .m, .jl, .R & SPSS,
Stata, Gretl, Minitab, Octave, MATLAB, Julia, R \\
\textbf{Resultados} & .spv, .smcl, .gph, .mtw, .txt & SPSS, Stata,
Gretl, Minitab \\
\textbf{Gráficos} & .png, .jpg, .bmp, .pdf, .fig, .wpg & Casi todos los
software \\
\textbf{Proyecto/Trabajo} & .mpj, .wf1 & Minitab, EViews \\
\textbf{Sesión} & .mtw & Minitab \\
\textbf{Modelo/Ecuación} & .equ, .sys & EViews \\
\textbf{Datos en formato binario} & .mat, .jld2, .RData & MATLAB, Julia,
R \\
\end{longtable}

\textbf{Explicación de las agrupaciones:}

\begin{itemize}
\tightlist
\item
  \textbf{Datos:} Almacenan los conjuntos de datos utilizados para el
  análisis.
\item
  \textbf{Sintaxis/Script:} Contienen los comandos o código que se
  ejecutan para realizar los análisis.
\item
  \textbf{Resultados:} Guardan los resultados de los análisis, como
  tablas, estadísticas y resultados de modelos.
\item
  \textbf{Gráficos:} Almacenan las representaciones visuales de los
  datos o resultados.
\item
  \textbf{Proyecto/Trabajo:} Contienen toda la información relacionada
  con un proyecto específico, incluyendo datos, análisis y resultados.
\item
  \textbf{Sesión:} Registran la sesión de trabajo, incluyendo los
  comandos ejecutados y los resultados obtenidos.
\item
  \textbf{Modelo/Ecuación:} Almacenan modelos estadísticos o ecuaciones
  estimadas.
\item
  \textbf{Datos en formato binario:} Almacenan datos en un formato
  optimizado para el software, a menudo utilizado para grandes conjuntos
  de datos o estructuras de datos complejas.
\end{itemize}

\textbf{Observaciones adicionales:}

\begin{itemize}
\tightlist
\item
  \textbf{CSV (Comma Separated Values):} Es un formato de texto plano
  muy común para intercambiar datos entre diferentes programas.
\item
  \textbf{Excel (.xlsx):} Aunque no está diseñado específicamente para
  análisis estadístico, se utiliza comúnmente para almacenar y manipular
  datos.
\item
  \textbf{Otros formatos:} Existen muchos otros formatos de archivo que
  pueden ser utilizados, dependiendo del software y la tarea específica.
\end{itemize}

\textbf{¿Necesitas más detalles sobre alguna extensión en particular o
sobre cómo convertir entre diferentes formatos?}

\textbf{Ejemplo de uso:}

Si tienes un archivo de datos en formato .csv y quieres analizarlo en
SPSS, puedes importarlo a un nuevo archivo .sav. Luego, escribirías un
script .sps para realizar el análisis y guardar los resultados en un
archivo .spv. Finalmente, podrías exportar los gráficos generados en
formato .png para incluirlos en un informe.

\textbf{¿Hay algo más que te gustaría saber sobre los archivos en
software estadístico?}

¡Claro! Aquí tienes la misma opción en inglés, pero con algunas palabras
más concisas y comunes en el ámbito de la programación y el análisis de
datos:

\textbf{Option 2: More descriptive words:}

\begin{itemize}
\tightlist
\item
  \textbf{Datos:} Data
\item
  \textbf{Sintaxis/Script:} Code
\item
  \textbf{Resultados:} Output
\item
  \textbf{Gráficos:} Graphs
\item
  \textbf{Proyecto:} Project
\item
  \textbf{Sesión:} Session
\item
  \textbf{Modelo:} Model
\item
  \textbf{Binario:} Binary
\end{itemize}

\textbf{A continuación te explicamos por qué se suelen utilizar estos
términos:}

\begin{itemize}
\tightlist
\item
  \textbf{Datos:} Este es el término más directo para la información en
  bruto con la que estás trabajando.
\item
  \textbf{Código:} Se refiere a las instrucciones que das al ordenador
  para realizar cálculos o crear visualizaciones.
\item
  \textbf{Salida:} Es el resultado de ejecutar tu código, como tablas,
  gráficos o resultados numéricos.
\item
  \textbf{Gráficos:} Este es un término más general para cualquier
  representación visual de tus datos.
\item
  \textbf{Proyecto:} Se refiere a la colección general de archivos y
  código relacionados con un análisis específico.
\item
  \textbf{Sesión:} Es un registro de las acciones que realizaste durante
  un periodo de trabajo específico.
\item
  \textbf{Modelo:} Se refiere a una representación matemática de un
  fenómeno del mundo real.
\item
  \textbf{Binario:} Se refiere a los datos almacenados en un formato
  binario, que a menudo se utiliza para la eficiencia.
\end{itemize}

\textbf{¿Quieres que te dé ejemplos de cómo utilizar estos términos en
los nombres de los archivos?}

Traducción realizada con la versión gratuita del traductor DeepL.com

\section{Publicaciones Similares}\label{publicaciones-similares}

Si te interesó este artículo, te recomendamos que explores otros blogs y
recursos relacionados que pueden ampliar tus conocimientos. Aquí te dejo
algunas sugerencias:

\begin{enumerate}
\def\labelenumi{\arabic{enumi}.}
\tightlist
\item
  \href{https://achalmaedison.netlify.app/programacion-software/fundamentos-programacion/2018-06-09-typografia/index.pdf}{\faIcon{file-pdf}}
  \href{https://achalmaedison.netlify.app/programacion-software/fundamentos-programacion/2018-06-09-typografia}{Typografia}
\item
  \href{https://achalmaedison.netlify.app/programacion-software/fundamentos-programacion/2019-08-13-organizando-tipos-de-archivos-de-softwares/index.pdf}{\faIcon{file-pdf}}
  \href{https://achalmaedison.netlify.app/programacion-software/fundamentos-programacion/2019-08-13-organizando-tipos-de-archivos-de-softwares}{Organizando
  Tipos De Archivos De Softwares}
\item
  \href{https://achalmaedison.netlify.app/programacion-software/fundamentos-programacion/2021-01-04-01-pensamiento-logico/index.pdf}{\faIcon{file-pdf}}
  \href{https://achalmaedison.netlify.app/programacion-software/fundamentos-programacion/2021-01-04-01-pensamiento-logico}{01
  Pensamiento Logico}
\item
  \href{https://achalmaedison.netlify.app/programacion-software/fundamentos-programacion/2021-01-11-02-introduccion-a-los-algoritmos/index.pdf}{\faIcon{file-pdf}}
  \href{https://achalmaedison.netlify.app/programacion-software/fundamentos-programacion/2021-01-11-02-introduccion-a-los-algoritmos}{02
  Introduccion A Los Algoritmos}
\item
  \href{https://achalmaedison.netlify.app/programacion-software/fundamentos-programacion/2021-01-18-03-programacion-basica-en-html/index.pdf}{\faIcon{file-pdf}}
  \href{https://achalmaedison.netlify.app/programacion-software/fundamentos-programacion/2021-01-18-03-programacion-basica-en-html}{03
  Programacion Basica En Html}
\item
  \href{https://achalmaedison.netlify.app/programacion-software/fundamentos-programacion/2024-03-31-por-editar/index.pdf}{\faIcon{file-pdf}}
  \href{https://achalmaedison.netlify.app/programacion-software/fundamentos-programacion/2024-03-31-por-editar}{Por
  Editar}
\end{enumerate}

Esperamos que encuentres estas publicaciones igualmente interesantes y
útiles. ¡Disfruta de la lectura!






\end{document}
